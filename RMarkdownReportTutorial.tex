\documentclass[]{article}
\usepackage{lmodern}
\usepackage{amssymb,amsmath}
\usepackage{ifxetex,ifluatex}
\usepackage{fixltx2e} % provides \textsubscript
\ifnum 0\ifxetex 1\fi\ifluatex 1\fi=0 % if pdftex
  \usepackage[T1]{fontenc}
  \usepackage[utf8]{inputenc}
\else % if luatex or xelatex
  \ifxetex
    \usepackage{mathspec}
  \else
    \usepackage{fontspec}
  \fi
  \defaultfontfeatures{Ligatures=TeX,Scale=MatchLowercase}
\fi
% use upquote if available, for straight quotes in verbatim environments
\IfFileExists{upquote.sty}{\usepackage{upquote}}{}
% use microtype if available
\IfFileExists{microtype.sty}{%
\usepackage[]{microtype}
\UseMicrotypeSet[protrusion]{basicmath} % disable protrusion for tt fonts
}{}
\PassOptionsToPackage{hyphens}{url} % url is loaded by hyperref
\usepackage[unicode=true]{hyperref}
\hypersetup{
            pdftitle={Salmon smolts: answers to all remaining questions},
            pdfauthor={Daniella and Sean},
            pdfborder={0 0 0},
            breaklinks=true}
\urlstyle{same}  % don't use monospace font for urls
\usepackage[margin=1in]{geometry}
\usepackage{graphicx,grffile}
\makeatletter
\def\maxwidth{\ifdim\Gin@nat@width>\linewidth\linewidth\else\Gin@nat@width\fi}
\def\maxheight{\ifdim\Gin@nat@height>\textheight\textheight\else\Gin@nat@height\fi}
\makeatother
% Scale images if necessary, so that they will not overflow the page
% margins by default, and it is still possible to overwrite the defaults
% using explicit options in \includegraphics[width, height, ...]{}
\setkeys{Gin}{width=\maxwidth,height=\maxheight,keepaspectratio}
\IfFileExists{parskip.sty}{%
\usepackage{parskip}
}{% else
\setlength{\parindent}{0pt}
\setlength{\parskip}{6pt plus 2pt minus 1pt}
}
\setlength{\emergencystretch}{3em}  % prevent overfull lines
\providecommand{\tightlist}{%
  \setlength{\itemsep}{0pt}\setlength{\parskip}{0pt}}
\setcounter{secnumdepth}{0}
% Redefines (sub)paragraphs to behave more like sections
\ifx\paragraph\undefined\else
\let\oldparagraph\paragraph
\renewcommand{\paragraph}[1]{\oldparagraph{#1}\mbox{}}
\fi
\ifx\subparagraph\undefined\else
\let\oldsubparagraph\subparagraph
\renewcommand{\subparagraph}[1]{\oldsubparagraph{#1}\mbox{}}
\fi

% set default figure placement to htbp
\makeatletter
\def\fps@figure{htbp}
\makeatother

\usepackage{lineno}
\linenumbers
\usepackage{setspace}\doublespacing
\setlength{\parskip}{1em}

\title{Salmon smolts: answers to all remaining questions}
\author{Daniella and Sean}
\date{}

\begin{document}
\maketitle

\section{Abstract}\label{abstract}

Salmon smolts are really cool, obviously. However, we also lack a full
understanding of why they do what they do. Here we fill in that
knowledge gap with some data, which we manipulate, analyze, and display
in R markdown. Buckle up\ldots{}

\section{Introduction}\label{introduction}

We will start out broad and reference some important contributions to
the literature. For example, you can reference a paper at the end of a
sentence like this (Naman et al. 2014). Alternatively, you may want to
refer to a specific study in the text. For example, LoScerbo et al.
(2020) is Daniella's recent paper - not about salmon, but still looks
really cool. To cite multiple references, separate them with a
semi-colon (Bailey and Moore 2020; LoScerbo et al. 2020).

\section{Methods}\label{methods}

If you are doing anything mathy, writing in Rmarkdown is great. You can
either use markdown or LaTeX syntax for equations. Simple math is easy
in markdown - just stick it between two dollar signs. You can mix LaTeX
syntax in as well for more complexity.

\(E = MC^2\)

You can also write maths with LaTeX directly. This is handy if you want
equations to be auto-numbered. Below is Bayes' theorem as LaTeX.

\begin{equation}
P(A \mid B) = \frac{P(B \mid A) \, P(A)}{P(B)}
\end{equation}

\section{Figures}\label{figures}

This figure is pretty lit \includegraphics{yourfirstfig.jpeg}

\section{References}\label{references}

\hypertarget{refs}{}
\hypertarget{ref-Bailey2020}{}
Bailey, C. J., and Jonathon W. Moore. 2020. ``Resource pulses increase
the diversity of successful competitors in a multi-species stream fish
assemblage.'' \emph{Ecosphere} 11 (September).
doi:\href{https://doi.org/10.1002/ecs2.3211}{10.1002/ecs2.3211}.

\hypertarget{ref-LoScerbo2020}{}
LoScerbo, Daniella, Maxwell J. Farrell, Julie Arrowsmith, Julia
Mlynarek, and Jean Philippe Lessard. 2020. ``Phylogenetically conserved
host traits and local abiotic conditions jointly drive the geography of
parasite intensity.'' \emph{Functional Ecology}, no. September: 1--11.
doi:\href{https://doi.org/10.1111/1365-2435.13698}{10.1111/1365-2435.13698}.

\hypertarget{ref-Naman2014}{}
Naman, S M, P M Kiffney, G R Pess, T W Buehrens, and T R Bennett. 2014.
``Abundance and body condition of sculpin (Cottus spp.) in a small
forest stream following recolonization by juvenile coho salmon.''
\emph{River Research and Applications} 30: 360--71.
doi:\href{https://doi.org/10.1002/rra}{10.1002/rra}.

\end{document}
